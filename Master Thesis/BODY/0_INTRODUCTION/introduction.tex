% INTRODUCTION FILE
\pagestyle{fancy}
\chapter*{Introduction}
\pagestyle{fancy}
The exponential growth of unstructured data, such as text, images, and audio, has outpaced the capabilities of traditional relational databases, which are optimized for structured records and ACID compliance. While conventional databases excel at transactional consistency, they struggle with efficient retrieval and similarity search across high-dimensional data spaces. This challenge has led to the rise of \textbf{Vector Database Management Systems (VDBMS)}, which represent data as vector embeddings, enabling powerful search and retrieval mechanisms based on similarity rather than exact matching.  

VDBMS leverage Approximate Nearest Neighbors (ANN) search to efficiently query large-scale datasets, balancing speed and accuracy. Unlike traditional indexing methods, vector databases employ structures such as \textbf{Hierarchical Navigable Small World (HNSW) graphs} and \textbf{Inverted File Index (IVF)} to optimize search performance. Additionally, quantization techniques reduce memory and storage footprints while preserving retrieval quality.  

Beyond retrieval, the application of VDBMS extends to \textbf{Multi-Modal search}, where different data types—such as images, text, and audio—are mapped into a unified vector space for seamless cross-modal querying. Furthermore, the integration of VDBMS with \textbf{Retrieval-Augmented Generation (RAG)} has gained traction, enhancing Large Language Model (LLM) responses by retrieving relevant context from vectorized data sources.  

This thesis explores the architecture, indexing techniques, and retrieval strategies of modern VDBMS. It evaluates leading implementations such as Weaviate, FAISS, and Milvus, comparing their performance in large-scale search applications. Additionally, we investigate the role of vector databases in \textbf{Agentic RAG}, examining strategies to mitigate issues improve retrieval precision through knowledge graphs and optimized prompting techniques.  

This work explores the capabilities and limitations of Vector Database Management Systems (VDBMS) to provide a comprehensive understanding of their impact on AI-driven search and retrieval. We highlight their role as a foundational technology in modern information systems. Finally, we introduce and analyze GraphRAG, a novel retrieval technique, comparing it to vector-based RAG. We examine their respective applications, advantages, and limitations, illustrating different use cases. As a final experiment, we propose an architecture for join discovery, leveraging knowledge graphs as a knowledge base to link entities and tables, enabling users to query structured data using natural language prompts.
